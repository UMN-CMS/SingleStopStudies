\documentclass[10pt]{beamer}
\usepackage[headerslides]{umnslides}
\usepackage{tikz}
\usetikzlibrary{fit,positioning,calc,shapes,backgrounds}
\usetikzlibrary{shapes.geometric, arrows}



\newcommand{\commonfiles}[1]{../common/#1}
\usepackage[subpreambles=true]{standalone}



\author{Charlie Kapsiak}
\title{Non-parametric 2D Background Estimation Using Gaussian Processes}
\begin{document}
\begin{frame}
  \maketitle
\end{frame}

\begin{frame}{Table Of Contents}
  \tableofcontents
\end{frame}

% \section{Introduction}
% \label{sec:introduction}
% 
% \begin{frame}{Bump Hunting}
%   \begin{itemize}
%   \item One of the most common classes of searches for new physics are \textit{bump hunts}.
%     \item Given some sort of smooth description of the standard model background, look for a localized bump of excess events coming from a new physics process.
%     \item 
%   \end{itemize}
% 
%   Why bbkg est is important
% \end{frame}
% 
% \begin{frame}{Background Estimation Strats}
%   General types of background estimation
% \end{frame}
% 
% 
% \begin{frame}{Ad-Hoc}
%   Drawbacks of ad-hoc parametric
% \end{frame}
% 
% \begin{frame}{Glimpse of GP}
%   Introduce idea of GP
% \end{frame}

\section{Introduction}

\begin{frame}{Signal Model}
  \begin{itemize}
  \item Searching for then production and decay of a single \stopq{} to 4 standard model quarks through an RPV coupling. 
  \item Well motivated channel to look for SUSY:
    \begin{itemize}
    \item Unexplored region of RPV parameter space
    \item Large cross section allows us to probe higher masses
    \end{itemize}
  \end{itemize}


  \begin{center}
    \scalebox{0.7}{\includestandalone{\commonfiles{general/single_stop}}}
  \end{center}
\end{frame}

\begin{frame}{General Search Strategy}
  \begin{itemize}
  \item General search strategy is to perform a one or two dimensional bump hunt for both the \stopq{} and the \chargino{} resonances. 
  \item Depending on mass splitting the resonances may be well separated both in $m_{\stopq, reco}$ and $m_{\chargino, reco}$ space, providing additional discriminating power.
  \item Key point is to effectively estimate the background. 
  \item However, a simple cut strategy on one mass axis can result in sculpting of the background, making estimation difficult. 
  \end{itemize}
\end{frame}

\begin{frame}{Estimation Strategies}
  \begin{itemize}
    \item For all bump hunts, key technique is estimation of the background shape. 
    \item Traditional bump hunts have used ad-hoc functions, chosen because they approximate the observed shape. 
    \item However, this can introduce bias from the choice of function, and it has been shown that they may scale poorly with increasing luminosity. 
    \item For multidimensional searches, the problem can also be compounded by selecting a function for a potentially nontrivial 2D shape. 
  \end{itemize}
\end{frame}

\begin{frame}{Current Strategy}
  \begin{itemize}
    \item We have implemented our background estimation using Gaussian process regression.
    \item This is non-parametric technique that reduces the bias from the choise of parametric function
    \item It has been shown to be robust against increasing luminosity.
    \item It is naturally extensible to multiple dimensions.
    \item Very well studied in statistics literature, and has a large number of well established implementations. 
  \end{itemize}
\end{frame}




\section{Gaussian Process Regression Overview}
\label{sec:gauss-proc-regr}

\begin{frame}{What is a Gaussian Process?}
  \begin{definition}
    A gaussian process is a possibly infinite series of random variables, any finite subset of which is jointly gaussian.
  \end{definition}


  Generall, the random variables are indexed by real values $x$, since we are generally considering regression over $\mathbb{R}^{n}$.

  A gaussian process $f(x)$ is is completely defined by its mean and covariance
  \begin{equation}
    \begin{split}
      m(x) &= \mathbb{E} \left[ f(x) \right] \\
      k(x,x') &= \mathbb{E} \left[ \left( f(x) - m(x) \right)  \left( f(x') - m(x') \right)\right] 
    \end{split}
  \end{equation}

  \begin{alertenv}
    Generally $m(x)$ is assumed to be 0, since this amounts to a shift, and generally the regression deduces the posterior mean very well. 
  \end{alertenv}
  
\end{frame}

\begin{frame}{Function Distributions}
  Gaussian process allow us to define distributions over the space of functions. Given a gaussian process $\mathcal{GP} \left( m(x) , k(x,x') \right)$, and some function $h$, then
  \begin{equation}
    \mathbf{h} \sim N(m(X) , k(X,X))
  \end{equation}

Given $n$ points in $\mathbb{R}^{k}$, the gaussian process defined a $n$ dimensional multivariate gaussian $\mathcal{N}$. If a function $h(x)$ has has values $h_1,h_2,...,h_{n}$ at those points, then
\begin{equation}
  p \left( h \right) \sim \mathcal{N}(h_{1}, ..., h_{n})
\end{equation}
\end{frame}




\begin{frame}{}
  
\end{frame}

\begin{frame}{}
  1-2 slides
  Discussion of parts of GP, specifically importance of the kernel.
\end{frame}


\section{2D Gaussian Processes and Kernels}
\label{sec:2d-gauss-proc}
\begin{frame}{Kernels}
  Importance of kernels, especially in 2D
\end{frame}

\begin{frame}{Discussion of Several Kernels}
  several slides outlining the different kernels 
\end{frame}

\section{Results for 1D and 2D Resonances}
\label{sec:results-1d-2d}
\begin{frame}{1D results}
  1-2 slides
  talk about kernel, show results
\end{frame}


\begin{frame}{2D Results}
  4-5 slides
  Present information on 2D fits, metrics, successes and challenges
\end{frame}


\section{Conclusion}
\label{sec:conclusion}

\begin{frame}{Ongoing Work}
Ongoing work
  
\end{frame}

\begin{frame}{Conclusion}

  Conclusion
  
\end{frame}



\end{document}


