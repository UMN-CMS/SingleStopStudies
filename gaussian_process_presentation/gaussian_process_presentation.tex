\documentclass[10pt]{beamer}
\usepackage[headerslides]{umnslides}
\usepackage{tikz}
\usetikzlibrary{fit,positioning,calc,shapes,backgrounds}
\usetikzlibrary{shapes.geometric, arrows}



\newcommand{\commonfiles}[1]{../common/#1}
\usepackage[subpreambles=true]{standalone}



\author{Charlie Kapsiak}
\title{Non-parametric 2D Background Estimation Using Gaussian Processes}
\begin{document}
\begin{frame}
  \maketitle
\end{frame}

\begin{frame}{Table Of Contents}
  \tableofcontents
\end{frame}

\begin{frame}{Overview}
  \includestandalone{\commonfiles{/gp/test}}
\end{frame}


% \section{Introduction}
% \label{sec:introduction}
% 
% \begin{frame}{Bump Hunting}
%   \begin{itemize}
%   \item One of the most common classes of searches for new physics are \textit{bump hunts}.
%     \item Given some sort of smooth description of the standard model background, look for a localized bump of excess events coming from a new physics process.
%     \item 
%   \end{itemize}
% 
%   Why bbkg est is important
% \end{frame}
% 
% \begin{frame}{Background Estimation Strats}
%   General types of background estimation
% \end{frame}
% 
% 
% \begin{frame}{Ad-Hoc}
%   Drawbacks of ad-hoc parametric
% \end{frame}
% 
% \begin{frame}{Glimpse of GP}
%   Introduce idea of GP
% \end{frame}

\section{Introduction}

\begin{frame}{Signal Model}
  \begin{itemize}
  \item Searching for then production and decay of a single \stopq{} to 4 standard model quarks through an RPV coupling. 
  \item Well motivated channel to look for SUSY:
    \begin{itemize}
    \item Unexplored region of RPV parameter space
    \item Large cross section allows us to probe higher masses
    \end{itemize}
  \end{itemize}


  \begin{center}
    \scalebox{0.7}{\includestandalone{\commonfiles{general/single_stop}}}
  \end{center}
\end{frame}

\begin{frame}{General Search Strategy}
  \begin{itemize}
  \item General search strategy is to perform a one or two dimensional bump hunt for both the \stopq{} and the \chargino{} resonances. 
  \item Depending on mass splitting the resonances may be well separated both in $m_{\stopq, reco}$ and $m_{\chargino, reco}$ space.
  \item As with all bump hunts, key point is to effectively estimate the background. 
  \end{itemize}
\end{frame}

\section{Gaussian Process Regression Overview}
\label{sec:gauss-proc-regr}

\begin{frame}{}
  3-4 slides
  Slides here explaining what a GP is, how it allows for inference over functions themselves. 
\end{frame}

\begin{frame}{}
  1 slides
   Quick 1D examples, maybe not even necessarily from this experiment.  
\end{frame}

\begin{frame}{}
  1-2 slides
  Discussion of parts of GP, specifically importance of the kernel.
\end{frame}


\section{2D Gaussian Processes and Kernels}
\label{sec:2d-gauss-proc}
\begin{frame}{Kernels}
  Importance of kernels, especially in 2D
\end{frame}

\begin{frame}{Discussion of Several Kernels}
  several slides outlining the different kernels 
\end{frame}

\section{Results for 1D and 2D Resonances}
\label{sec:results-1d-2d}
\begin{frame}{1D results}
  1-2 slides
  talk about kernel, show results
\end{frame}


\begin{frame}{2D Results}
  4-5 slides
  Present information on 2D fits, metrics, successes and challenges
\end{frame}


\section{Conclusion}
\label{sec:conclusion}

\begin{frame}{Ongoing Work}
Ongoing work
  
\end{frame}

\begin{frame}{Conclusion}

  Conclusion
  
\end{frame}



\end{document}


