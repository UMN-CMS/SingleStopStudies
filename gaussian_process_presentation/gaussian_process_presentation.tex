\documentclass[10pt]{beamer}
\usepackage{umnslides}
\usepackage{tikz}
\usetikzlibrary{fit,positioning,calc,shapes,backgrounds}
\usetikzlibrary{shapes.geometric, arrows}


\author{Charlie Kapsiak}
\title{Non-parametric 2D Background Estimation Using Gaussian Processes}
\begin{document}
\begin{frame}
  \maketitle
  
\end{frame}

\begin{frame}{Overview}

\end{frame}

\section{Introduction}
\label{sec:introduction}


\begin{frame}{Bump Hunting}
  General class of searches are bump hunts. What is a bump hunt.

  Why bbkg est is important
\end{frame}

\begin{frame}{Background Estimation Strats}
  General types of background estimation
\end{frame}


\begin{frame}{Ad-Hoc}
  Drawbacks of ad-hoc parametric
\end{frame}

\begin{frame}{Glimpse of GP}
  Introduce idea of GP
\end{frame}

\section{Gaussian Process Regression Overview}
\label{sec:gauss-proc-regr}

\begin{frame}{}
  3-4 slides
  Slides here explaining what a GP is, how it allows for inference over functions themselves. 
\end{frame}

\begin{frame}{}
  1 slides
   Quick 1D examples, maybe not even necessarily from this experiment.  
\end{frame}

\begin{frame}{}
  1-2 slides
  Discussion of parts of GP, specifically importance of the kernel.
\end{frame}


\section{2D Gaussian Processes and Kernels}
\label{sec:2d-gauss-proc}
\begin{frame}{Kernels}
  Importance of kernels, especially in 2D
\end{frame}

\begin{frame}{Discussion of Several Kernels}
  several slides outlining the different kernels 
\end{frame}

\section{Results for 1D and 2D Resonances}
\label{sec:results-1d-2d}
\begin{frame}{1D results}
  1-2 slides
  talk about kernel, show results
\end{frame}


\begin{frame}{2D Results}
  4-5 slides
  Present information on 2D fits, metrics, successes and challenges
\end{frame}


\section{Conclusion}
\label{sec:conclusion}

\begin{frame}{Ongoing Work}
Ongoing work
  
\end{frame}

\begin{frame}{Conclusion}

  Conclusion
  
\end{frame}



\end{document}


